\paragraph*{Jetzt zu zeigen} Falls $f(k) \in PKP \Rightarrow k \in MPKP$

\para{} Sei $(i_1,\dots,i_n)$ eine Lösung für die PKP-Instanz $f(k)$.

\paragraph*{Beobachtung 1} Es gilt $i_1=0$ und $i_n=k+1$ da nur $x_0'$ und $y_0'$ mit dem selben Zeichen anfangen und $x_{k+1}'$ und $y_{k+1}'$ mit demselben Zeichen enden.

\paragraph*{Beobachtung 2} Es gilt $\forall_j=2,\dots,n; i_j \not=0$, d.h. die 0-te Karte taucht nur an erster Stelle auf, da sobald 0-te Karte gelegt wird, oben \#\# auftaucht, was unten nicht erzeugbar ist.

\paragraph*{Beobachtung 3} Es gilt $\forall_j=2,\dots,n-1; i_j \not= k+1$, d.h. das $(k+1)$-te Kärtchen taucht nur an letzter Stelle auf, da wir sonst eine kürzere Lösung des PKP finden könnten (nämlich bis zum ersten Auftreten des $(k+1)$-ten Kärtchens).

\para{} Daraus folgt, dass die PKP-Lösung für $f(k)$ folgende Struktur hat: $$ x_0'x_{i_2}'\dots x_{i_n}' = \#a_1\#a_2\dots\#a_s\#\$ = y_0'y_{i_2}'\dots y_{i_n}' $$ für entsprechende Zeichen $a_l \in \Sigma$. Die MPKP-Instanz $k$ hat somit die Lösung: $$ x_1x_{i_2}x_{i_3}\dots x_{i_{n-1}} = a_1,\dots,a_s = y_1y_{i_2}y_{i_3}\dots y_{i_{n-1}} $$
\begin{itemize}
	\item[$\Rightarrow$] $f(k) \in PKP \Rightarrow k \in MPKP$
	\item[$\Rightarrow$] $k \in MPKP \Leftrightarrow f(k) \in PKP$ $\Box$
\end{itemize}
