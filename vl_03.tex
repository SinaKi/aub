Zeige, dass die Wahrscheinlichkeit, dass Kargers MinCut Algorithmus das richtige Ergebnis berechnet, mindestens $\frac{1}{n^2}$ ist. Das klingt nicht besonders gut, aber wir können die Erfolgswahrscheinlichkeit beliebig erhöhen durch Wiederholung.

\begin{lstlisting}[mathescape]
RepMinCut(G,r)
	bestVal = $\infty$, bestSol = $\emptyset$
	for i = 1 to r do
		p = KArgerMinCut(G)
		if val(P) < bestVal
			bestVal = val(P)
			bestSol = P
		fi
	od
	return bestSol
\end{lstlisting}
\para{$p$ berechnet den Cut vom Orginalalgorithmus. $val(P)$ ist der Wert aus Cuts.}

\para{Angenommen Kargers MinCut berechnet MinCut mit $WS \geq \frac{1}{n^2}$. Die Wahrscheinlichkeit, dass in jedem der $r$ Versuche \underline{nicht} der MinCut zu finden ist, ist $\big(1-\frac{2}{n(n-1)}\big)^r < \big(e^{-\frac{2}{n(n-1)}}\big)^r$ oder auch für $r \approx n^2 \log n \Rightarrow $ Fehler $WS \leq \frac{1}{n^c}$.}

\para{$e^{-\frac{2}{n(n-1)}}$ ist die obere Schranke für die WS, dass man jedes Mal Pech hat z.B. $r=\frac{n(n-1)}{2} \Rightarrow$ WS, dass falsch $\leq \frac{1}{e} \approx 0,36$.}

\paragraph*{Beweise,} dass die Erfolgswahrscheinlichkeit in der Tat $\geq \frac{2}{n(n-1)}$ ist. Wir fixieren hierzu einen MinCut und zeigen, dass die Wahrscheinlichkeit, nie eine Kante des MinCut zu Kontrahieren $\geq \frac{2}{n(n-1)}$ ist.
Die Wahrscheinlichkeit im ersten Schritt eine MinCut zu bekommen ist $\frac{k}{\#Kanten}$.

\paragraph*{Lemma:} Betrachte einen Multigraph $G(V,E)$ mit einem MinCut mit Wert $k$. Dann gilt: $G$ hat mindestens $\frac{k \cdot n}{2}$ Kanten.
\paragraph*{Beweis:} Beobachtung: Jeder Knoten hat mindestens Grad $k$. $\Rightarrow$ Anzahl der Kanten $\geq k \cdot \frac{n}{2}$.

\para{Sei $E_i \dots$ Ergebnis, dass in i-tem Kantenkontraktionsschritt keine MinCut-Kante erwischt wurde. Wir betrachten einen fixen MinCut mit Wert $k$ (bzw. dessen $k$ Kanten). Die Wahrscheinlichkeit, im ersten Kontraktionsschritt einen MinCut zu erwischen ist $\frac{k}{m} \leq \frac{2}{n} \textcolor{gray}{= \big( m \geq \frac{k \cdot n}{2} \big)} \Rightarrow Pr(E_1) \geq 1-\frac{2}{n}$.}
\para{Falls $E_1$ eingetreten ist, existieren vor dem zweiten Schritt midestens $\frac{k \cdot (n-1)}{2}$ Kanten.}
\para{$\Rightarrow$ Wahrscheinlichkeit, im zweiten Schritt eine MinCut Kante zu erwischen (unter der Voraussetzung $E_1$ ist eingetreten) ist
$$ \leq \frac{k}{\frac{k(n-1)}{2}} = \frac{2}{n-1}$$
$$ \Rightarrow Pr(E_2|E_1) \geq 1 - \frac{2}{n-1} $$}
\para{Falls $E_1$ und $E_2$ eingetreten sind, existieren vor dem dritten Schritt mindestens $\frac{k(n-2)}{2}$ Kanten.}
\para{$\Rightarrow$ Wahrscheinlichkeit, im dritten Schritt eine MinCut Kante zu erwischen (unter der Voraussetzung, dass $E_1 \& E_2$) eingetreten sind) ist
$$ \leq \frac{k}{\frac{n(n-2)}{2}} = \frac{2}{n-2} $$
$$ \Rightarrow Pr(E_3|E_1 \cap E_2) \geq 1 - \frac{2}{n-2} $$}

\paragraph*{Allgemein:} Vor dem i-ten Schritt haben wir noch $n-i+1$ Kanten. Falls wir bislang  noch keinen Fehler gemacht haben ($E_1,\dots, E_{n-1}$ sind eingetreten) hat der MinCut noch Größe $k$.
$$ \Rightarrow \text{ wir haben } \geq k(n-i+1) \text{ Kanten} $$
$$ \Rightarrow Pr(E_i|\bigcap_{j=1}^{i-1} E_j) \geq 1-\frac{2}{n-i+1} $$
Uns interessiert $Pr(\bigcap\limits_{i=1}^{n-2} E_i)$
$$ = Pr(E_1) \cdot Pr(E_2|E_1) \cdot Pr(E_3|E_1 \cap E_2) \cdot Pr(E_4|E_1 \cap E_2 \cap E_3) \cdots \geq \prod\limits_{i=1}^{n-2} \frac{n-i-1}{n-i+1} $$
$$ = \frac{n-2}{n} \cdot \frac{n-3}{n-1} \cdot \frac{n-4}{n-2} \cdot \frac{n-5}{n-3} \cdot \frac{n-6}{n-4} \cdots \frac{2}{4} \cdot \frac{1}{3} $$
$$ = \frac{2}{n(n-1)} $$

$\Rightarrow$ Pr(Algorithmus berechnet wirklich MinCut) $\geq \frac{2}{n(n-1)}$
\para{Durch Wiederholung lässt sich die Erfolgswahrscheinlichkeit erhöhen. Wenn wir z.B. $c \cdot \frac{n^2}{2} \log n$ oft wiederholen erhalten wir eine Fehlerwahrscheinlichkeit $\leq \frac{1}{n^c}$. 'Der Algorithmus hat eine hohe Erfolgswahrscheinlichkeit'.}
\para{Falls die Fehlerwahrscheinlichkeit $\leq \frac{1}{2^n}$ ist, sagen wir 'Der Algorithmus hat eine sehr hohe Erfolgswahrscheinlichkeit'. Hier können wir das für $~ c \cdot \frac{n^2}{2} \cdot n$ Wiederholungen erreichen.}

\begin{table}
\centering
\begin{tabular}{l|l}
closest Pair & MinCut \\
\hline
\hline
liefert \underline{immer} korrektes Resultat & kann falsches Resultat liefern, abhängig vom Zufall \\
\hline
Laufzeit hängt vom Zufall ab & Laufzeit fix \\
& \\
\textcolor{blue}{$\Rightarrow$LAS VEGAS ALGORITHMUS} & \textcolor{blue}{$\Rightarrow$ MONTE CARLO ALGORITHMUS} \\
\end{tabular}
\end{table}

\paragraph*{Geg.:} LAS VEGAS ALGORITHMUS mit erwarteter Laufzeit $E(T) = f(n)$ ist es möglich daraus einen Monte Carlo Algorithmus zu bauen? Ja, und zwar wie folgt:
\minisec{} Stoppe Las Vegas Algorithmus nach $\alpha \cdot f(n)$ Zeit für Konstantes $\alpha > 1$. Falls Las Vegas Algorithmus bis dahin terminiert hat, ist das Ergebnis nicht korrekt. Algorithmus macht Fehler, wenn Zeit ausgeht. \\
$\Rightarrow$ müssen Wahrscheinlichkeit berechnen, dass Las Vegas Algorithmus länger als $\alpha \cdot f(n)$ Zeit braucht.

\paragraph*{Markov-Ungleichung} Sei $X$ eine nicht-negative Zufallsvariable mit $E(X) = \gamma$.
Dann gilt: $Pr(X > \alpha \cdot \gamma) \leq \frac{1}{\alpha}$ \\
$\Rightarrow$ in unserem modifizierten Algorithmus, der Las Vegas Algorithmus nach $\alpha \cdot f(n)$ Zeit abbricht, bekommen wir eine Fehlerwahrscheinlichkeit $\leq \frac{1}{\alpha}$.

\paragraph*{Las Vegas $\Rightarrow$ Monte Carlo} \dots immer möglich.
\paragraph*{Monte Carlo $\Rightarrow$ Las Vegas} \dots im Allgemeinen nicht möglich. Nur mit Hilfsmitteln.
\paragraph*{Annahme} Monte Carlo Algorithmus mit fixer Laufzeit $f(n)$ mit Erfolgswahrscheinlichkeit $p(n)$ und zusätzlich einen Checker-Algorithmus C mit Laufzeit $g(n)$, welcher die Korrektheit eines Ergebnisses überprüfen kann. \\
$\Rightarrow$ daraus ergibt sich folgender Las Vegas Algorithmus:
\begin{lstlisting}
lasse A laufen (Kosten: f(n))
ueberpruefe Ergebnis (Kosten: g(n))
falls falsch wiederholen
\end{lstlisting}

Pro Runde haben wir Laufzeit $f(n)+g(n)$. Die erwartete Anzahl Runden ist $\frac{1}{p(n)}$. \\
$\Rightarrow$ die erwartete Laufzeit dieses Las Vegas Algorithmus ist $\frac{f(n)+g(n)}{p(n)}$.

\section{Quicksort}
weiterer Las Vegas Algorithmus Quicksort.
\begin{lstlisting}[mathescape]
RQS(A[1,$\dots$,n])
	waehle p $\in$ $\{$ 1,$\dots$,n $\}$
	rearrangiere A sodass
		$\underbrace{<A[p]}_{A_L}$ A[p] $\underbrace{>A[p]}_{A_R}$
		RQS($A_L$)
		RQS($A_R$)
\end{lstlisting}

\paragraph*{Anmerkung} Wird $p$ deterministisch sehr simpel wie z.B. $p01,p=\lfloor\frac{n}{2}\rfloor,\dots$ gewählt, dann gibt es immer Instanzen die Quicksort zu $~n^2$ Schritten zwingen.
Wir wählen $p$ zufällig und zeigen, dass die erwartete Laufzeit $O(n \log n)$ ist. Das heißt insbesondere, dass wir keine bösartige Eingabe erzeugen können, die Quicksort immer lang laufen lässt.