\section{Randomisierte Algorithmen}
Ein randomisierter Algorithmus ist ein Algorithmus, welcher unter Nutzung einer Zufallsquelle ein Problem löst. Oft sind randomisierte Algorithmen deutlich einfacher und teilweise auch effizienter als entsprechende deterministische Algorithmen. Typischerweise analysiert man Algorithmen bzgl. Platz- und Zeitbedarf. Randomisierte Algorithmen kann man auch bzgl. 'Verbrauch von Zufall' analysieren.


\subsection{Closest Pair}
\begin{itemize}
	\item[] Geg.: $n$ Punkte $P \in \mathbb{R}^2$
	\item[]Ges.: $p_1,p_2 \in P$ mit $|p_1 p_2|$ minimal
\end{itemize}


\subsubsection{Naive Lösung}
\begin{itemize}
	\item[] Betrachte $p_1$ und betrachte die Distanz zu $p_2,\dots,p_n$ merke Minimum. %TODO satz korrekt?
	\item[] Betrachte $p_2$ und betrachte Distanz zu $p_2,\dots,p_n$.
	\item[] \hspace{1cm}\vdots
	\item[] Betrachte $p_{n-1}$ und betrachte Distanz zu $p_n$.
\end{itemize}

\minisec{} Gebe Minimum aller gemessenen Distanzen aus.

\paragraph*{Aufwand} = Anzahl gemessener Distanzen = $n-1 + n-2 + \dots + 1 = \sum\limits_{i=1}^{n-1} i = \frac{(n-1)n}{2} = \Theta(n^2)$
 
\minisec{} Ist $\sim n^2$ Laufzeit gut oder schlecht?

\paragraph*{Bsp.} CPU mit 1GHz, $10^9$ Instruktionen/Sekunden

\paragraph*{Alg.} welcher $n^2$ Instruktionen für Eingabe der Größe braucht, läuft für
\begin{itemize}
	\item[] $n=100 \Rightarrow (100)^2 \cdot 10^{-9} = 10^{-5} \Rightarrow 10\mu s$
	\item[] $n=1000 \Rightarrow (1000)^2 \cdot 10^{-9} = 10^{-3} \Rightarrow 1ms$
	\item[] $n=100000 \Rightarrow (100000)^2 \cdot 10^{-9} = 10 \Rightarrow 10s$
	\item[] $n=1000000 \Rightarrow (10^6)^2 \cdot 10^{-9} = 10^3 \Rightarrow 1000s$ (>15min)
\end{itemize}

\minisec{} Closest Pair kann deterministisch in $O(n \log n)$ Zeit gelöscht werden. (devide \& conquer) %TODO komplexität richtig?
Man kann sogar beweisen, dass Closest Pair vergleichsbasiert nicht Schneller als $\Omega(n \log n)$ gelöst werden kann.

\paragraph*{Jetzt} Randomisierter Algorithmus, der Closest Pair in erwartet $O(n)$ Zeit löst.

\minisec{} $X$ sein eine Zufallsvariable. Der Erwartungswert einer diskreten Zufallsvariable ist $E(X)=\sum\limits_i \cdot Pr(x=i)$

\paragraph*{Bsp.} $X$ \dots ist die Augenzahl eines Wurfs mit einem Würfel. $X$ kann Werte $\{ 1,2,3,4,5,6 \}$ annehmen. $E(X) = \sum\limits_{i=1}^{6} i \cdot \underbrace{Pr(x=i)}_{\frac{1}{6}} = \sum\limits_{i=1}^{6} i \cdot \frac{1}{6} = \frac{1}{6} \cdot \sum\limits_{i=1}^{6} i \cdot = \frac{42}{62} = 3,5$

\minisec{} Wir entwerfen also einen Algorithmus, dessen Laufzeit als Zufallsvariable $X$ dargestellt werden kann. Bei Eingabe $n$ gilt $E(X) = O(n)$.


\subsection{Inkrementeller Algorithmus für Closest Pair}
Wir betrachten Punkte in Reihenfolge $p_1,p_2,\dots,p_n$. Sei $\delta_i$, die Closest Pair Distanz der Punktmenge $\{ p_1,p_2,\dots,p_i \}$. Angenommen wir kennen $\delta_i$, wie können wir $\delta_{i+1}$ bestimmen?

\paragraph*{Naiv} Vergleiche $p_{i+1}$ mit $p_1,p_2,\dots,p_i$ und setze $\delta_{i+1} = min(\delta_i, min|p_{i+1},p_j|)$ mit $j = 1,2,\dots,i$. $\Rightarrow$ Gesamtlaufzeit wieder $(1+2+\dots+(n+1)) = \Theta(n^2)$

\paragraph*{Verbesserung} Angenommen wir haben $\delta_i$ bestimmt und auch ein Gitter mit Maschenweite $\delta_i$ erzeugt, in welches alle Punkte $p_1,p_2,\dots,p_i$ eingeordnet sind.
%TODO bild

\paragraph*{Berechnung von $\delta_{i+1}$}
\begin{itemize}
	\item lokalisiere $p_{i+1}$ im Gitter
	\item inspiziere nur Punkte in benachbarten Gitterzellen von $p_{i+1}$ (und eigenen Zelle)
	\item falls neue Closest Pair Distanz, setze $\delta_{i+1}$ entsprechend, ansonsten $\delta_{i+1} = \delta_i$
\end{itemize}

\paragraph*{Lemma} In einer Gitterzelle liegen $\leq 4$ Punkte. Beweis Flächeninhalt von Punktkreisen.

\minisec{} Falls $\delta_i = \delta_{i+1}$, füge $p_{i+1}$ in Gitter ein. Falls $\delta_1 > \delta_{i+1}$, baue Gitter neu auf für neues $\delta_{i+1}$ (kostet $\Theta(i)$). Schlecht!

%TODO bild

\minisec{} Bei dieser Konfiguration und Einfügereihenfolge muss nach jedem Schritt das Gitter neu aufgebaut werden. $\Rightarrow$ Laufzeit wieder $\sum\limits_{i=1}^{n-1} \approx n^2$

\paragraph*{Abhilfe} Füge Punkte in zufälliger Reihenfolge ein. D.h. jede Permutation soll gleich Wahrscheinlich $\frac{1}{n!}$ auftreten. %TODO n! richtig?

\paragraph*{Zentrale Aussage} Wahrscheinlichkeit, dass $\delta_i < \delta_{i-1} \leq \frac{2}{i}$.

\paragraph*{Dann} Erwartete Kosten des Einfügens von $p_i \leq \frac{2}{i} \cdot O(i) + O(1) = O(1)$
\marginpar{\textcolor{blue}{\scriptsize $\frac{2}{i} \cdot O(i)$ Schlechter Fall, dass Gitter neu aufgebaut werden muss}}
Erwartete Gesamtlaufzeit $E$($\sum\limits_i$ kosten für Einfügen von $p_i$) =\footnote{Linearität des Erwartungswerts} $\sum\limits_i^n$ (kosten für Einfügen von $p_i$) = $\sum\limits_i^n O(1) = O(n)$






