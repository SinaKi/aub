\section{Das Wörterbuchproblem}

\paragraph*{Geg.} Universum $U$ (sehr groß); Teilmenge $S \subseteq U$ (mittelgroß) mit $|S|=n$; ein $m \in \mathbb{N}$

\paragraph*{Ziel} Finde: $h: U \rightarrow \{ 0,\dots,\dots m-1 \}$ sodass $\forall\ 0 \leq i \leq m\ \big| \{ x \in s | h(x)=i \} \big| \leq \frac{n}{m}$

\paragraph*{Bsp.} $U=\mathbb{N}$; $S=\{ 1,7,23,99 \}$ mit $n=4$ und $m=5$

Eine gute Funktion $h$ für $S$ wäre z.B. $h(x)=x \mod 5$, wobei $h$ die Hashfunktion ist.
\begin{table}[htb!]
\centering
\begin{tabular}{l|l}
$x$ & $h(x)$ \\ 
\hline 
1 & 1 \\ 
7 & 2 \\ 
23 & 3 \\ 
99 & 4 \\ 
\end{tabular} 
\end{table}

Die gleiche Hashfunktion ist für ein $S = \{ 12,22,17,32,52 \}$ sehr schlecht. Alle Elemente fallen in eine Kategorie.


\subsection{Anwendungsbeispiele}

\subsubsection{Telefonbuch}
\para{Man möchte für Name + Vornamedie zugehörige Telefonnummer herausfinden.}

\para{$U \overset{\wedge}{=}$ Menge aller Zahlen die irgendeiner Name-Vorname-Kombination entsprechen.}

\para{$S \leq U \overset{\wedge}{=}$ Name-Vorname in Stuttgart. Mit $|n| = 500000$ und $m = 500 000$}

\para{Könnten wir eine Hashfunktion: $h:U \rightarrow \{ 0,\dots m-1 \}$ abbildet mit $\forall 0 \leq i < m ; \big| \{ x\in S | h(x)=i \} \big| \leq \frac{n}{m} = 1$} %TODO ; richtig

\para{Das ist Toll!}

\para{Wir können ein Array anlegen mit 500000 Einträgen. Die Telefonnummer eines 'Name+Vorname' speichern wir an positiven $h$(Name+Vorname) des Arrays.}

\subsubsection{Stowasser (Latein-Deutsch Wörterbuch)}
$U$ \dots Menge aller Wörter. $S$ \dots Menge aller lateinischen Wörter.

\subsubsection{randomisierter Closest Pair Algorithmus}
$U$ \dots Menge aller Paare $(i,j)$, $i \in \mathbb{Z}$, $j \in \mathbb{Z}$. $S$ \dots Menge der Paare $(i,j)$, für die in entsprechender Gitterzelle mindestens ein Punkt liegt.

\subsubsection{Open-Street-Map-Daten}
Basiert auf Knoten und Kanten. Anzahl der Knoten ungefähr 4-5 Milliarden. $U \dots \mathbb{N}$. $S$ \dots Menge der KnotenIDs im aktuellen Kartenausschnitt.

\para{Bei guter Hashfunktion $h$ könnte man in $O(1)$ Informationen mit $S$ assoziieren/auslesen.}

\subsubsection{Datenstruktur Beispiel}
Wir betrachten ein Wörterbuch als abstrakte Datenstruktur. Diese sollte folgende Operationen unterstützen:
\begin{itemize}
	\item[] S = MAKESET() \dots leeres Wörterbuch erzeugen
	\item[] insert(x,S) \dots fügt Item $x$ (Item besteht aus (k,inf)) in $S$ ein bzw. falls Item mit Schlüssel $k$ existiert, ersetze diesen.
	\note{z.B. Schlüssel: Name+Vorname; Info: Tel.nr.}
	\item[] delete(x,S) \dots löscht Item $x$ aus Wörterbuch
	\item[] lookup(k,S) \dots gibt Item $x = (k,inf)$ aus, falls vorhanden
	\note{$k$ eindeutig}
\end{itemize}


\subsection{Naive Implementierung}

\subsubsection*{1. Array/Feld für die Items, Zähler für Anzahl Items in $S$.}
%TODO bild
\begin{itemize}
	\item[] insert(x,S) \dots überprüfen ob Schlüssel von $x$ schon enthalten, falls ja, überschreiben des entsprechenden Items, falls nein, Item hinten anhängen und Zähler eins hoch. Laufzeit: $\Theta(n)$
	\item[]  delete(x,S) \dots Item finden, austauschen mit letztem Item, Zähler eins runter. Laufzeit: $\Theta(n)$
	\item[] lookup(k,S) \dots alle Items durchgehen, Item mit passendem Schlüssel zurückgeben. Laufzeit: $O(1)$
	\item[] MAKESET \dots
\end{itemize}
Vorteile: Einfach, Platzsparend

\subsubsection*{2. Verkettete Liste (einfach oder doppelt)}
 \dots ähnlich, aber braucht nicht unbedingt zusammenhängenden Speicher
 
\subsubsection*{3. Direkte Adressierung}
\paragraph*{Annahme} Schlüsseluniversum endlich, $[0,\dots,k-1]$. Wir spendieren ein Array der Größe $k$. An Position $i$ des Arrays steht das Item mit Schlüssel $i$ (falls vorhanden) oder Sonderwert 'NIL' falls keiner vorhanden. Insert/Delete/lookup haben alle Zeitverbrauch $O(1)$.

\paragraph*{Problem} Platzverbrauch \textasciitilde Größe des Schlüsseluniversums.
\note{z.B. bei der Verketteten Liste war der Platzverbrauch \textasciitilde \# zu verwendende Elemente}

\paragraph*{Ziel} Datenstruktur für Wörterbuch, welche
\begin{itemize}
	\item O(1) Zugriffszeit für insert/delete/lookup hat
	\item O(n) Platzverbrauch garantiert
\end{itemize}

\para{Idealerweise hätten wir gerne für ein gegebenes $S \subseteq U = \{ 0,\dots,2^{64}-1 \}$ mit $|S|=n$ eine Hashfunktion $h$, welche $S$ injektiv auf $\{ 0,\dots,n-1 \}$ abbildet, d.h. $\forall x,y \in S: h(x) \not= h(y)$.}


\para{Mit einem solchen $h$ könnten wir $S$ wie folgt verwalten.}
\begin{itemize}
	\item lege Array mit $n$, Einträgen für die Items an
	\item Item mit Schlüssel $x$ wird an Position $h(x)$ gespeichert. (an keiner Stelle werden $>1$ Items gespeichert)
	\item lookup(x,S): Schaue an Position $h(x)$ nach
\end{itemize}

\paragraph*{Vorteil}
\begin{itemize}
	\item Zugriffszeit $O(1)$
	\item Größe $O(|S|)$
\end{itemize}

\para{$U$ \dots Schlüsseluniversum}

\para{$T[0,\dots,t-1]$ \dots Feld, auch Hashtafel genannt}

\para{$S \subseteq U$ \dots Teilmenge von Schlüsseluniversum, welche gehasht werden soll}

\paragraph*{Gesucht} $h:U \rightarrow [0,\dots,t-1]$

\para{$c_x^h(S) := \{ y \in S | h(y) = h(x) \}$ \dots die Menge an elementen in $S$, die von $h$ auf den selben Hashfeldeintrag gemappt werden wie $x$}

\para{Idealerweise $|c_x^h(S) \leq 1|$ ($\Rightarrow h$ injektiv für $S$). Falls $t<|S|$, hätte man gerne $|c_x^h(S)| \leq \lceil\frac{|S|}{t}\rceil$}

\paragraph*{Frage} Gibt es eine Superhashfunktion $h^*$, welche für jedes $S \subseteq U$ garantiert, dass $$ |c_x^h(S)| \leq \lceil \frac{|S|}{t} \rceil $$

\paragraph*{Bsp.} $U=\mathbb{N}$; $S=\{ 20,13,6,18,28,313 \}$; $t=5$; $h(x)= x \mod 5$. Für $S=\{ 11,33,22,19,5 \}$ wäre dieses $h$ perfekt. Für $S=\{ 16,51,26,31,1 \}$ ist dieses $h$ Katastrophal.
%TODO hashtabelle

\paragraph*{Satz} Seien $U,t$ und $h$ gegeben, $|U|=h$. Dann gilt es für alle $n$ mit $1 \leq \frac{h}{t}$ ein $S \subseteq U$ mit $|S|=n$ (schreibe auch $S \in {U \choose n} $) mit  $$ |c_x^h(S)| = n \forall x \in S, $$ d.h., alle Elemente aus $S$ werden in den gleichen Hashtafeleintrag gehasht.

\paragraph*{Beweis} Nach Schubfachprinzip existiert ein $i$ mit $0 \leq i < t$ sodass $$ |\underbrace{h^{-1}(i)}_{\{ x \in U | h(x)=i \}}| \geq \frac{|U|}{t} $$ $\Rightarrow \frac{|U|}{t} = \frac{h}{t} \geq n$. Wähle $S$ aus ${h^{-1}(i) \choose n}$ \hspace{8cm} $\Box$

\para{$\Rightarrow$ Es gibt also keine Superhashfunktion die für jedes $S$ gut ist; man sollte also die Hashfunktion $h$ randomisiert oder unter Berücksichtigung von $S$ bestimmen}