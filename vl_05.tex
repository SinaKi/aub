\para{Interpretiere jeden Anlauf, den Algorithmus berechnen zu lassen als einen Münzwurf mit
\begin{itemize}
	\item[] Kopf $\overset{\wedge}{=}$ Algorithmus wurde fertig
	\item[] Zahl $\overset{\wedge}{=}$ Algorithmus wurde nicht fertig
	\item[$\Rightarrow$] Pr(Kopf) $\geq 1 - \frac{1}{k}$
	\item[] Pr(Zahl) $\leq \frac{1}{k}$
\end{itemize}
}

\paragraph*{Uns interessiert} also, wie oft man eine Münze $(a-\frac{1}{k};\frac{1}{k})$ werfen muss, bis Kopf kommt.

\paragraph*{Ergebnis} Erwartet müssen wir $\frac{k}{k-1}$ mal werfen.

\paragraph*{Noch interessanter} Pr(mehr als $m$ Münzwürfe nötig bis Kopf kommt) $= (\frac{1}{k})^m$

\paragraph*{Bsp.}
\begin{itemize}
	\item[] Pr(Kopf) $= 1 - \frac{1}{k}$ mit $m = \log n$ %TODO wo kommt das m her?
	\item[] Pr(Zahl) $= \frac{1}{k}$ mit $k=2$
	\item[$\Rightarrow$] Die Wahrscheinlichkeit, dass wir mehr als $\log_2 n$ mal werfen müssen, bis Kopf kommt ist $\leq (\frac{1}{2})^{\log_2 n} = \frac{1}{n}$
\end{itemize}

\paragraph*{$\Rightarrow$} Mit hoher Wahrscheinlichkeit braucht Random Quicksort mit Wraper
\begin{itemize}
	\item[] $\leq 2n (\log n)(\log n)$
	\item[] $\leq 2n (\log_2 n)$ Schritte
\end{itemize}
also $Pr(X > 2n \log^2 n)  \leq \frac{1}{n}$

\paragraph*{Zum Vergleich} Mit Markov Ungleichung können wir zeigen, dass $$ Pr(X > 2n \log^2 n) \leq \frac{1}{2 \log n} $$

\paragraph*{Genauso gilt für Wraper-Algorithmus} $$ Pr(X > 2 \cdot 5 \cdot n \cdot \log^n) \leq \frac{1}{n^5} $$


\section{Zero-knowledge-Proof}
\paragraph*{Anwendung} Online-Banking. Sie gehen auf die Webseite Ihrer Bank; Wie können Sie sicher sein, auf deren Webseite zu sein und nicht auf dem MockUp Ihres WG-Genossen, der ihre TAN abgreifen will?

\paragraph*{Ansatz} Die Bank soll Ihnen beweisen, dass Sie ein Geheimnis kennt, welches nur die Bank (uns Sie) kennen kann, allerdings ohne das Geheimnis zu verraten.

\paragraph*{In Cryptosprache} Alice möchte Bob beweisen, dass Sie ein Geheimnis kennt, ohne dass Bob etwas über das Geheimnis lernt (Zero-knowledge-Proof).

\subsection{Gewünschte Eigenschaften des Protokolls}
\begin{itemize}
	\item Falls Alice das Geheimnis nicht kennt, wird Sie mit hoher Wahrscheinlichkeit von Bob ertappt.
	\item Alice gibt nichtr preis, was Bob nicht sowiso schon weiß.
\end{itemize}
Das Protokoll basiert auf dem Graphisomorphproblem.

\subsection{Graphisomorphproblem}

\paragraph*{Geg.} Graphen $G_1(V_1,E_1)$ und $G_2(V_2,G_2)$

\paragraph*{Frage} Sind $G_1$ und $G_2$ isomorph?

\para{$G_1$ und $G_2$ sind isomorph zueinander, falls eine Bijektion $\varphi : V_1 \rightarrow V_2$ existiert, sodass $\forall (u,v) \in E_1 \Leftrightarrow (\varphi(u),\varphi(v)) \in E_2$}

\subsubsection{Beispiele}

\begin{figure}[!h]
\centering
\begin{tikzpicture}[->,>=stealth',shorten >=1pt,auto,node distance=2.5cm,
        thick,main node/.style={circle,draw,minimum size=1cm,inner sep=0pt]}]

	\node at (-2,1) {$G_1:$};
	\node at (3.7,1) {$G_2:$};

    \node[main node] (1) {3};
    \node[main node] (2) [below = 1.5cm of 1] {4};
    \node[main node] (3) [below right = 0.7cm and 1cm of 1] {2};
    \node[main node] (4) [below left = 0.7cm and 1cm of 1] {1};
    \node[main node] (5) [right = 4.5cm of 1] {B};
    \node[main node] (6) [below = 1.5cm of 5] {D};
    \node[main node] (7) [below right = 0.7cm and 1cm of 5] {C};
    \node[main node] (8) [below left = 0.7cm and 1cm of 5] {A};

    \path[-]
    (1) edge (3)
        edge (4)
    (2) edge (4)
    	edge (3)
    (3) edge (4)
    (5) edge (6)
        edge (7)
        edge (8)
    (6) edge (7)
    	edge (8);
\end{tikzpicture}
\end{figure}

Isomorph, Beweis:
\begin{tabular}{l|l}
	v & $\varphi$ \\
	\hline
	1 & D \\
	2 & B \\
	3 & A \\
	4 & C \\
\end{tabular}

\begin{figure}[!h]
\centering
\begin{tikzpicture}[->,>=stealth',shorten >=1pt,auto,node distance=2.5cm,
        thick,main node/.style={circle,draw,minimum size=1cm,inner sep=0pt]}]

	\node at (-2,1) {$G_1:$};
	\node at (3.7,1) {$G_2:$};

    \node[main node] (1) {1};
    \node[main node] (2) [right = 1cm of 1] {2};
    \node[main node] (3) [below = 0.5cm of 2] {3};
    \node[main node] (4) [right = 4cm of 1] {A};
    \node[main node] (5) [right = 1cm of 4] {B};
    \node[main node] (6) [below = 0.5cm of 5] {C};

    \path[-]
    (1) edge (2)
    (4) edge (5)
    (5) edge (6);
\end{tikzpicture}
\end{figure}
Nicht isomorph, da identische Anzahl Knoten und Kanten notwendig sind für die Existenz einer Isomorphie.

\newpage %TODO bessere lösung um graph richtig zu positionieren
\begin{figure}[!h]
\centering
\begin{tikzpicture}[->,>=stealth',shorten >=1pt,auto,node distance=2.5cm,
        thick,main node/.style={circle,draw,minimum size=1cm,inner sep=0pt]}]

	\node at (-2,1) {$G_1:$};
	\node at (3.7,1) {$G_2:$};

    \node[main node] (1) {1};
    \node[main node] (2) [right = 1.7cm of 1] {2};
    \node[main node] (3) [below right = 0.5cm and 0.7cm of 1] {4};
    \node[main node] (4) [below left = 2.3cm and 0.7cm of 2] {3};
    \node[main node] (5) [right = 5cm of 1] {A};
    \node[main node] (6) [below left = 0.7cm and 0.7cm of 5] {B};
    \node[main node] (7) [below right = 0.7cm and 0.7cm of 5] {C};
    \node[main node] (8) [below = 2cm of 5] {D};
    
    \path[-]
    (1) edge (2)
    	edge (4)
    (2) edge (3)
    (2) edge (4)
    (3) edge (4)
    (5) edge (6)
    	edge (7)
    (8) edge (6)
    	edge (7)
    (6) edge (7);
    
\end{tikzpicture}
\end{figure}

\begin{tabular}{l|l}
	v & $\varphi$ \\
	\hline
	1 & A \\
	2 & C \\
	3 & B \\
	4 & D \\
\end{tabular}

%TODO graph
Nicht isomorph, da $G_2$ einen Grad 4 Knoten enthält und $G_1$ nicht.

\para{Bislang gibt es keinen Polynomzeitalgorithmus, der Graphisomorphie entscheidet. NP-Harte ist nicht bekannt.}

\subsection{Geheimnis der Bank/Alice} Isomorphismus zwischen zwei öffentlichen bekannten Graphen $G_1$ und $G_2$. Bank (Alice) möchte Sie (Bob) davon überzeugen, dass sie $\varphi$ kennt, ohne $\varphi$ zu verraten.

\subsection{Setup (nur 1 mal)}
Alice erzeugt z.B. zufällig einen riesigen Graph $G_1$  und durch zufällige Permutation der Knotenmenge (das ist der geheime Isomorphismus $\varphi$) auch ein isomorpher Graph $G_2$. $G_1$ und $G_2$ werden öffentlich gemacht, $\varphi$ bleibt das Geheimnis von Alice (Bank).

\subsection{Protokoll}
Protokoll, durch welches Alice Bob überzeugt, das Geheimnis zu kennen.

\para{Alice permutiert $G_j$ (mit $j \in \{ 1,2 \}$ zufällig) mit einer zufälligen Permutation $\Pi$ zu $H$ (und stellt sicher, dass $H \not= G_1,G_2$) und schickt $H$ zu Bob.}

\paragraph*{Bob} Möchte sich davon überzeugen, dass Alice Isomorphie zwischen $G_1$ und $G_2$ kennt, indem er $k \in \{ 1,2 \}$ zufällig wählt und Alice bittet, die Isomorphie zwischen $G_k$ und $H$ offenzulegen.

\paragraph*{Alice} Zeigt entweder $\Pi$ (falls $j=k$) oder $\Pi \circ \varphi$ (bzw. $\Pi \circ \varphi^{-1}$).

%TODO graphen

\para{Falls Alice $\varphi$ kennt, kann Sie immer korrekt antworten. Falls Alice $\varphi$ nicht kennt und $H$ aus $G_1$ konstruiert hat, kann Sie Bobs Anfrage nach Isomorphie zwischen $G_1$ und $H$ einfach beantworten, allerdings wird es Alice schwer fallen eine Isomorphie zwischen $G_2$ und $H$ zurückzugeben, es sei denn sie kann das Graphisomorphieproblem lösen.}

\paragraph*{Theorem} Alice verrät nichts über $\varphi$.

\paragraph*{Beweis} Bob lernt Isomorphie z.B. $G_1$ und einer zufälligen Permutation von $G_1$. Diese Isomorphie hätte er sich selber basteln können.

\paragraph*{Theorem} Bob bekommt probabilistischen Beweis, dass Alice $\varphi$ kennt. Bei f-maliger Wiederholung ist die Wahrscheinlichkeit, dass Bob immer den Graph $G_j$ wählt, den Alice zur Erzeugung des jeweiligen $H$ verwendet hat $= \frac{1}{2^f}$