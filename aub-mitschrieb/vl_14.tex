\subsection{Vergleich zwischen TM und RAM}

\paragraph*{Behauptung} RAM und Turingmaschine sind gleichmächtig.

\par\medskip Offensichtlich ist RAM mindestens so mächtig wie die Turingmaschine. Nicht ganz so offensichtlich ist, dass die Turingmaschine mindestens so mächtig ist wie die RAM.\par\medskip

\paragraph*{Satz} Jede $t(n)$ zeitbeschränkte RAM kann durch eine $O(q(n+t(n)))$-zeitbeschränkte Turingmaschine simuliert werden für ein Polynom $q()$.

\paragraph*{Uns interessiert nun die Frage} Was kann man (mit Rechnern von Heute) berechnen, was ist prinzipiell nicht berechenbar?

\par\medskip Wir werden dazu über die Turingmaschine argumentieren, von denen wir uns überzeugt haben, dass sie gleichmächtig wie Registermaschinen (und 'echte' aktuelle Rechner) sind.\par\medskip

\subsection{Church-Turing-These}

Die Klasse der turingberechenbaren Funktionen stimmt mit der Klasse der Intuitiv berechenbaren Funktionen überein.

\section{Berechen-/Entscheidbarkeit}
\paragraph*{Def} Eine Menge $M$ heißt abzählbar, wenn es eine surjektive Funktion $c_\mathbb{N}\rightarrow M$ gibt. Nicht abzählbare Mengen heißen überabzählbar.

\paragraph*{Bsp.} für abzählbare Mengen
\begin{itemize}
	\item die Menge $\mathbb{Z}$ $$ c(i)= \begin{cases}\frac{i}{2} & \text{falls i gerade}\\ \frac{-(i+1)}{2} & \text{falls i ungerade}\end{cases} $$ %TODO bild
	\item die Menge der Wörter über $\{0,1\}^*$ jedes Wort über $\{0,1\}$ kann als Binärzahl aufgefasst werden (mit signifikantem Bit hinten) $01111_2=30_{10}$
	\item die Menge der Turingmaschinen ist abzählbar, da sie eine Teilmenge aller Wörter über $\{0,1\}$ ist (wenn z.B. Kodiert wie für universelle TM-Gödelnr.)
\end{itemize}

\paragraph*{Satz} Die Menge $P(\mathbb{N})$ \note{$P(\mathbb{N})$ Menge aller Teilmengen aus $\mathbb{N}$} ist überabzählbar.

\paragraph*{Beweis} Annahme $P(\mathbb{N})$ sei abzählbar, sei dabei $s_i$ die i-te Teilmenge gemäß dieser Nummerierung. Definiere folgende Matrix($A_{i,j}$) mit $i\in\mathbb{N},j\in\mathbb{N}$ mit $$ A_{i,j} = \begin{cases}1&\text{falls } j\in s_i \\ 0 & \text{sonst}\end{cases} $$ \note{i: Zeile; j: Spalte}
%TODO bild

\par\medskip Basierend auf dieser Matrix definieren wir die Teilmenge $S_{DIAG}=\{ i\in\mathbb{N} \| A_{i,j}=1 \}$. Das Komplement davon ist $$ \overline{S_{DIAG}}=\{i\in\mathbb{N} \| A_{i,j}=0\} $$ Auch $\overline{S_{DIAG}}$ sollte in der Menge der Teilmengen auftreten. Angenommen $\overline{S_{DIAG}} = S_K$. Betrachte die Fälle \par\medskip
\begin{enumerate}
	\item Falls $A_{K,K}=1 \Rightarrow K \in S_L = \overline{S_{DIAG}}$ aber: $\overline{S_{DIAG}}$ ist definiert als Menge $\{i\in\mathbb{N}|A_{i,j}=0\} \lightning$ 
	\item Falls $A_{K,K}=0 \Rightarrow K \not\in S_K = \overline{S_{DIAG}}$ aber: $\overline{S_{DIAG}}$ ist definiert als Menge $\{i\in\mathbb{N}|A_{i,j}=0\} \lightning$
\end{enumerate}

\paragraph*{Beobachtung} Die Menge $\{0,1\}^*$ hat dieselbe Mächtigkeit wie $\mathbb{N}$, somit hat die Menge aller Sprachen über $\{0,1\}$ die selbe Mächtigkeit wie $P(\mathbb{N})$.

\par\medskip An dieser Stelle ist eigentlich schon klar, dass man nicht jede Sprache mit einer Turingmaschine entscheiden kann, da es überabzählbar viele Sprachen gibt, aber nur abzählbar viele Turingmaschinen.

\paragraph*{Konkretes Bsp.} für eine nicht-entscheidbare Sprache. Sei $w_i$ das i-te Wort in der Aufzählung aller Wörter über $\{0,1\}$. Sei $M_i$ die i-te Turingmaschine in der Aufzählung aller Turingmaschinen. Betrachte folgende Sprache $$ \text{Diagonalsprache } D=\{ w\in\{0,1\}^* \| w=w_i \text{ und } M_i \text{ akzeptiert } w \text{ nicht} \} $$ Also ein $w$ ist in der Sprache $D$ genau dann wenn $w$ das i-te Wort ist und von der i-ten Turingmaschine nicht akzeptiert wird.

\par\medskip Wie vorher könnte man diese Sprache mit Hilfe einer Matrix darstellen\par\medskip
\begin{table}[htb!]
\centering
\begin{tabular}{c|c c c c c c}
 & $w_0$ & $w_1$ & $w_2$ & $w_3$ & \dots & \\
\hline
$M_0$ & 0 & 1 & 1 & 1 & 0 & 1 \\
$M_1$ & 1 & 1 & 0 & 0 & 1 &  \\
$M_2$ & 0 & $A_{i,j}$ & 1 & 0 &  &  \\
\vdots & 1 & 0 & 1 & 0 &  &  \\
 & 1 & 1 & 0 & 1 &  &  \\
\end{tabular}
\end{table}
$$ A_{i,j} = \begin{cases}1&\text{falls } M_i\ w_i \text{ akzeptiert} \\ 0 & \text{sonst}\end{cases} $$
$$ D=\{ w_i \| A_{ii}=0 \}$$

\paragraph*{Satz} $D$ ist nicht Turing-entscheidbar.

\paragraph*{Beweis} Angenommen, es gibt eine Turingmaschine $M_j$, die $D$ entscheidet. Wir wenden $M_j$ auf das j-te Wort $w_j$ an.
\begin{description}
	\item[Fall 1] Falls $w_j \in D$, muss $M_j$ $w_j$ akzeptieren $\Rightarrow A_{j,j}=1 \lightning$ $A_{j,j}=0$ gemäß Definition von $D$.
	\item[Fall 2] Falls $w_j \not\in D$, damit $M_j$ $w_j$ nicht akzeptieren $\Rightarrow A_{j,j}=0 \lightning$ $A_{j,j}=1$ gemäß Definition von $D$. $\Box$
\end{description}

\paragraph*{Lemma} Sei $L$ eine unentscheidbare Sprache. Dann ist auch $\overline{L}$ (das Komplement von $L$) nicht entscheidbar.

\paragraph*{Beweis} Annahme, es gibt eine Turingmaschine $M_{\overline{L}}$, welche $\overline{L}$ entscheidet, d.h. auf jede Eingabe $w$ hält und genau dann akzeptiert, falls $w \in \overline{L}$. Wir konstruieren eine Turingmaschine $M$, welche $L$ entscheidet (und $M_{\overline{L}}$ als Unterprogramm benutzt). Maschine $M$ startet $M_{\overline{L}}$ auf $w$. Falls $M_{\overline{L}}$ akzeptiert, verwirft $M$, falls $M_{\overline{L}}$ verwirft, akzeptiert $M$. $M$ entscheidet somit $L \lightning$ zu Annahme, dass $L$ unentscheidbar. $\Box$


\subsection{Das Halteproblem}
Entscheidung, ob ein Programm auf eine bestimmte Eingabe terminiert. Formal: $$ H=\{\kod{M} w \| M \text{ hält auf Eingabe } w \} $$

\paragraph*{Satz} Das Halteproblem $H$ ist nicht entscheidbar.

\paragraph*{Beweis} Für Widerspruchsbeweis nehme an, dass die Turingmaschine $M_H$ $H$ entscheidet. Wir zeigen, dass wir mit $M_H$ als Unterprogramm eine Turingmaschine $M_{\overline{D}}$ konstruieren könnten welche $\overline{D}$ entscheidet. Das ist aber unmöglich, daher kann es $M_H$ nicht geben.

\begin{itemize}
	\item[$M_{\overline{D}}$:] \begin{enumerate}
		\item Für $w$ berechne $i$, sodass $w_i=w$
		\item Berechne die i-te  Turingmaschine $\kod{M_i}$ \note{$\kod{?}$ Kodierung von ?}
		\item Starte $M_H$ mit Eingabe $\kod{M_i}w$
		\item Falls $M_H$ akzeptiert, so simuliere das Verhalten von $M_i$ auf $w$ (mit universeller Turingmaschine) und übernehme das Ergebnis.
		\item Falls $M_H$ verwirft, verwirf Eingabe.
	\end{enumerate}
\end{itemize}

\paragraph*{Zur Korrektheit dieser Konstruktion} Sei $w=w_i$
\begin{itemize}
	\item[$w\in\overline{D}$] $\Rightarrow M_H$ akzeptiert $\kod{M_i}w$ und $M_i$ akzeptiert $w$
	\item[] $\Rightarrow M_{\overline{D}}$ akzeptiert $w$
	\item[$w\not\in\overline{D}$] $\Rightarrow M_H$ verwirft $\kod{M_i}w$ oder $M_i$ verwirft $w$
	\item[] $\Rightarrow M_{\overline{D}}$ verwirft $w$ \hspace{1cm} $\Box$
\end{itemize}