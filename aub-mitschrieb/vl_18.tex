\paragraph*{Jetzt zu zeigen} Falls $f(k) \in PKP \Rightarrow k \in MPKP$

\par\medskip Sei $(i_1,\dots,i_n)$ eine Lösung für die PKP-Instanz $f(k)$.

\paragraph*{Beobachtung 1} Es gilt $i_1=0$ und $i_n=k+1$ da nur $x_0'$ und $y_0'$ mit dem selben Zeichen anfangen und $x_{k+1}'$ und $y_{k+1}'$ mit demselben Zeichen enden.

\paragraph*{Beobachtung 2} Es gilt $\forall_j=2,\dots,n; i_j \not=0$, d.h. die 0-te Karte taucht nur an erster Stelle auf, da sobald 0-te Karte gelegt wird, oben \#\# auftaucht, was unten nicht erzeugbar ist.

\paragraph*{Beobachtung 3} Es gilt $\forall_j=2,\dots,n-1; i_j \not= k+1$, d.h. das $(k+1)$-te Kärtchen taucht nur an letzter Stelle auf, da wir sonst eine kürzere Lösung des PKP finden könnten (nämlich bis zum ersten Auftreten des $(k+1)$-ten Kärtchens).

\par\medskip Daraus folgt, dass die PKP-Lösung für $f(k)$ folgende Struktur hat: $$ x_0'x_{i_2}'\dots x_{i_n}' = \#a_1\#a_2\dots\#a_s\#\$ = y_0'y_{i_2}'\dots y_{i_n}' $$ für entsprechende Zeichen $a_l \in \Sigma$. Die MPKP-Instanz $k$ hat somit die Lösung: $$ x_1x_{i_2}x_{i_3}\dots x_{i_{n-1}} = a_1,\dots,a_s = y_1y_{i_2}y_{i_3}\dots y_{i_{n-1}} $$
\begin{itemize}
	\item[$\Rightarrow$] $f(k) \in PKP \Rightarrow k \in MPKP$
	\item[$\Rightarrow$] $k \in MPKP \Leftrightarrow f(k) \in PKP$ $\Box$
\end{itemize}

\note{Konfig.: Konfiguration einer Turingmaschine}

\begin{table}[htb!]
\begin{tabular}{l|l l l l l}
oben: & & Konfig. 1 & Konfig. 2 & Konfig. 3 & \dots \\
unten: & Konfig. 1 & Konfig. 2 & Konfig. 3 & Konfig. 4 & \dots \\
\end{tabular}
\end{table}

\paragraph*{Noch zu zeigen} $H \leq MPKP$ (Lemma 2.32)

\paragraph*{Idee} Simuliere durch Kärtchen gültige Berechnung der Turingmaschine.

\paragraph*{Bsp.} Betrachte Turingmaschine $M$:

\begin{tabular}{lll}
\parbox{5cm}{
	\begin{itemize}
		\item[$\Sigma$] $\{ 0,1 \}$
		\item[$\Gamma$] $\{ 0,1,B \}$
	\end{itemize}
} &
\parbox{5cm}{
	\begin{itemize}
	\item[$Q$] $\{ q_0,q_1,q_2,\overline{q} \}$
	\item[$q_0$] Anfangszustand
	\end{itemize}
} &
\parbox{5cm}{
	\begin{itemize}
	\item[$\overline{q}$] Endzustand
	\item[$\delta$] Übergangsfunktionen
	\end{itemize}
}
\end{tabular}

\begin{table}[htb!]
\centering
\begin{tabular}{c|c c c}
$\delta$ & 0 & 1 & B \\
\hline
$q_0$ & $(q_0,0,R)$ & $(q_1,1,R)$ & $(\overline{q},1,N)$ \\
$q_1$ & $(q_2,0,R)$ & $(q_1,1,R)$ & $(\overline{q},1,N)$ \\
$q_2$ & $(q_2,0,R)$ & $(q_2,1,R)$ & $(q_2,B,R)$ \\
\end{tabular}
\end{table}

\paragraph*{Semantik} $M$ testet, ob Eingabe die Form $0î1^j;i,j \geq 0$ hat.

\par\medskip Die Berechnung einer Turingmaschine kann durch eine Folge von Konfigurationen beschrieben werden.

\paragraph*{Notation} <BandinhaltLinksDesKopfs><Zustand><BandinhaltAbKopfNachRechts>

\par\medskip Im Beispiel ist die Konfiguration für Eingabe $w=0011$ wie folgt $$ \underbrace{q_0 0011}_{\text{Startkonfig.}} \mapsto 0 q_0 011 \mapsto 00q_0 11 \mapsto 001q_1 1 \mapsto 0011q_1 \mapsto 0011\overline{q}1 $$

\note{Nächste Konfiguration muss durch Anwendung der Übergangsfunktion aus der vorherigen Konfig. entstehen.}
\paragraph*{Idee des Beweis} Erzeuge MPKP-Instanz welche die Berechnung der Turingmaschine $\overset{\wedge}{=}$ Sequenz gültiger Konfigurationen Simulieren.

\par\medskip Das Startkärtchen sieht wie folgt aus: $$\Big[ \frac{\#}{\#\#q_0 0011} \Big]$$ Unten steht die Startkonfiguration.

\par\medskip Es gibt des weiteren 'Kopierkärtchen' $$\Big[ \frac{0}{0} \Big], \Big[ \frac{1}{1} \Big], \Big[ \frac{B}{B} \Big], \Big[ \frac{\#}{\#} \Big]$$ und 'Übergangskärtchen' $$\Big[ \frac{q_0 0}{0q_0} \Big], \Big[ \frac{q_0 1}{1q_1} \Big], \Big[ \frac{q_0 B}{\overline{q}1} \Big], \Big[ \frac{q_1 0}{0q_2} \Big], \Big[ \frac{q_1 1}{1q_1} \Big], \Big[ \frac{q_1 B}{\overline{q}1} \Big], \Big[ \frac{q_2 0}{0q_2} \Big], \Big[ \frac{q_2 1}{1q_2} \Big], \Big[ \frac{q_2 B}{Bq_2} \Big], \Big[ \frac{q_0\#}{\overline{q}1\#} \Big], \Big[ \frac{q_1\#}{\overline{q}1\#} \Big]$$

\paragraph*{Beispiel} $$ \Big[ \frac{\#}{\#\#q_0 0011\#} \Big]\Big[ \frac{\#}{\#} \Big]\Big[ \frac{q_0 0}{0 q_0} \Big]\Big[ \frac{0}{0} \Big]\Big[ \frac{1}{1} \Big]\Big[ \frac{1}{1} \Big]\Big[ \frac{\#}{\#} \Big] $$
$$ \Big[ \frac{\#}{\#} \Big], \Big[ \frac{0}{0} \Big], \Big[ \frac{q_0 0}{0q_0} \Big], \Big[ \frac{1}{1} \Big], \Big[ \frac{1}{1} \Big], \Big[ \frac{\#}{\#} \Big] $$
$$ \Big[ \frac{\#}{\#} \Big], \Big[ \frac{0}{0} \Big], \Big[ \frac{0}{0} \Big], \Big[ \frac{q_0 1}{1q_1} \Big], \Big[ \frac{1}{1} \Big], \Big[ \frac{\#}{\#} \Big] $$
$$ \Big[ \frac{\#}{\#} \Big], \Big[ \frac{0}{0} \Big], \Big[ \frac{0}{0} \Big], \Big[ \frac{1}{1} \Big], \Big[ \frac{q_1 1}{1q_1} \Big], \Big[ \frac{\#}{\#} \Big] $$
$$ \Big[ \frac{\#}{\#} \Big], \Big[ \frac{0}{0} \Big], \Big[ \frac{0}{0} \Big], \Big[ \frac{1}{1} \Big], \Big[ \frac{1}{1} \Big], \Big[ \frac{q_1\#}{\overline{q}1\#} \Big] $$
Die Berechnung oben hängt einen Berechnungsschritt hinterher, folgende 'Aufräumkärtchen' gleichen die Strings unten aus: $$ \Big[ \frac{\overline{q}0}{\overline{q}} \Big], \Big[ \frac{\overline{q}1}{\overline{q}} \Big], \Big[ \frac{\overline{q}B}{\overline{q}} \Big], \Big[ \frac{0\overline{q}}{\overline{q}} \Big], \Big[ \frac{1\overline{q}}{\overline{q}} \Big], \Big[ \frac{B\overline{q}}{\overline{q}} \Big] $$ sowie $$\Big[ \frac{\#\overline{q}\#\#}{\#} \Big]$$
$$ \Big[ \frac{\#}{\#} \Big], \Big[ \frac{0}{0} \Big], \Big[ \frac{0}{0} \Big], \Big[ \frac{1}{1} \Big], \Big[ \frac{1}{1} \Big], \Big[ \frac{\overline{q}1}{\overline{q}} \Big], \Big[ \frac{\#}{\#} \Big] $$
$$ \Big[ \frac{\#}{\#} \Big], \Big[ \frac{0\overline{q}}{\overline{q}} \Big], \Big[ \frac{\#}{\#} \Big], \Big[ \frac{\#\overline{q}\#\#}{\#} \Big] $$