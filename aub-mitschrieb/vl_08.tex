\paragraph*{Theorem} Es gibt keine 'immer gute' $(\forall S \subseteq U)$ Hashfunktion.

\paragraph*{Bsp.} $U=\{ 0,1,\dots,2^{64}-1 \}$ $S=\{ 3,7,9,10 \}$ $T=$ %TODO T= diagramm
$h(x) = x \mod 5$ Für dieses $S$ ist $h$ sehr gut. Für $S=\{ 6,11,31,46 \}$ aber nicht!

\paragraph*{Problem} Wenn $h(x_1)=h(x_2)$ für $x_1 \not= x_2$.


\subsection{Hashing mit Verkettung}
Jede Tafelposition ist Kopf einer Verketteten Liste. Alle $x \in S$ mit $h(x)=i$ werden in i-ter Liste gespeichert. Platzbedarf: $O(m+n) = O(n+\frac{1}{B})$ mit Belegungsfaktor $B=\frac{n}{m}$. %TODO notiz
Je kleiner $V$ desto platzineffizienter wird das Wörterbuch (aber dann evtl. einfacher, gute Hashfunktion zu finden).

\paragraph*{Zugriffszeit} (unter Annahme, dass $h()$ in $O(1)$ ausgewertet werden kann)
\begin{itemize}
	\item[] Zugriff auf $x \in S$ in O(1 + Position von $x$ in Liste $L_{h(x)}$)
	\item[] Zugriff auf $x \not\in S$ in O(1 + |$L_{h(x)}$|)
\end{itemize}

\paragraph*{Annahme} $h$ verteilt $U$ gleichmäßig über $T$, d.h. $\forall i \big|\underbrace{\{ x \in U | h(x)=i \}}_{U_i}\big| \leq \big|\frac{N}{m}\big|$

\paragraph*{Bsp.} $h(x) = x \mod m$

\paragraph*{Satz} Sei $x$ ein zufälliges Element aus $U-S$ und $n \leq \frac{N}{2}$. Dann ist die erwartete Suchzeit für $x$ $O(1+B)$.

\paragraph*{Beweis} $l_i =$ Anzahl Elemente aus $S$, die in $L_i$ gespeichert werden, $i=0,\dots,m-1$. Es gilt $\sum\limits_{i=0}^{m-1} l_i = n$. Die erwartete Suchzeit ist $$\bigg(\sum\limits_{i=0}^{m-1} Pr(h(x)=i) \cdot l_i\bigg)+1$$

\note{$U_i \subseteq U$ = Menge aus $U_i$ welche von $h$ auf $i$ gemappet wird}
\par\medskip
$Pr(h(x)=i) = \frac{|U_i \backslash S|}{|U \backslash S|} \leq \frac{|U_i|}{|U \backslash S|} \leq\lceil \frac{\frac{N}{m}}{\frac{N}{2}} \rceil \leq \frac{\frac{N}{m+1}}{\frac{N}{2}} = \frac{2}{m} + \frac{2}{N} \leq \frac{2}{m} + \frac{1}{n}$\par\medskip

$$\Rightarrow E = \big( \sum\limits_{i=0}^m Pr(h(x)=i) - l_i \big) + 1 \leq 1+ \sum\limits_{i=0}^{m-1}(\frac{2}{m}+\frac{1}{n}) - l_i = 1+\frac{2}{m} \cdot \underbrace{\sum l_i}_{n} + \frac{1}{n} \underbrace{\sum l_i}_{n}$$ %TODO - oder mal?
$$= 1+\frac{2n}{m}+1 = 2+\frac{2n}{m} = O(1+\frac{n}{m}) = O(1+B) \hspace{5cm} \Box$$

\par\medskip
Das heißt für zufällige Elemente nicht aus $s$ ist erwartete Zugriffszeit OK.\par\medskip

\paragraph*{Problem} In der Praxis sind Zugriffe nicht zufällig.

\paragraph*{Satz} Sei $x$ ein zufälliges Element aus $S$. Dann ist die erwartete Zugriffszeit für $x$ $$ O\Bigg(1+\frac{1}{n} - \sum\limits_{i=0}^{m-1}\Big( \frac{l_i(l_i+1)}{2} \Big)\Bigg) $$

\paragraph*{Anmerkung} Satz nicht besonders aussagekräftig, da $\sum l_i = n$ fast nichts aussagt über $\sum \frac{l_i(l_i+1)}{2}$.

\paragraph*{Beweis} Wenn $x$ das $j$-te Element in seiner Liste ist, dann ist die Suchzeit $O(1+j)$. Aber ist die erwartete Suchzeit $$ O\Big(\frac{1}{n} \cdot \sum\limits_{i=0}^{m-1} \sum\limits_{j=1}^{l_i} (1+j)\Big) = O\Big(\frac{1}{n} \cdot \sum\limits_{i=j}^{m-1} \frac{l_i(l_i+1)}{2}\Big) \hspace{4cm} \Box$$

\paragraph*{Satz} Sei $S$ eine zufällige (gleichverteilt) Teilmenge aus $U$ der Größe $n$. Dann ist die erwartete Zugriffszeit auf ein zufälliges $x \in S$ $$ \leq 1 + B \cdot \frac{3}{2} \cdot e^B $$ (für nicht zu großes $B$, z.B. $B=1$, ist das $O(1)$)

\paragraph*{Beweis} nicht klausurrelevant

\par\medskip
Wieder unbefriedigend: In Anwendung ist weder $S \subseteq U$ zufällig noch greift man zufällig auf die Elemente in $S$ zu.\par\medskip


\subsection{Universelles Hashing}
Sei $H$ eine Menge von Funktionen von $U$ nach $\{ 0,1,\dots,m-1 \}$.

\paragraph*{Definition} Für $c>1$ heißt $H$ c-universell$^*$ falls für alle $x,y \in U, x\not=y$, gilt:
$$ \frac{\big|\{ h \in H : h(x) = h(y) \}\big|}{|H|} \leq \frac{c}{m} $$

\paragraph*{Satz} Für $a,b \in \{ 0,1,\dots,N-1 \}$, wobei N eine Primzahl ist, sei $$h_{a,b}: x \mapsto \big( (ax+b) \mod N \big) \mod m$$ Dann ist die Klasse $$ H=\{ h_{a,b}: 0 \leq a,b \leq N-1 \} $$ c-universell$^*$ mit $c=\frac{\lceil\frac{N}{m}\rceil}{\frac{N}{m}} \approx 1$.

\paragraph*{Beweis} siehe Übung.

\paragraph*{Satz} Benutzt man Hashing mit Verkettung und wählt $h \in H$ zufällig gleichverteilt (mit $H$ c-universell), dann ist die erwartete Zugriffszeit für zugriffe $O(1+c \cdot B)$ für beliebige Mengen $S \subseteq U, |S|=n$.

\paragraph*{Beweis} Zeit für Zugriff auf $x$ $$ \leq 1 + 
\underbrace{\# x\in S \text{ mit } h(x)=h(y)}_{\mathclap{\substack{\text{wir werden dafür Erwartungswert}\\ \text{ für zufälliges } h \in H \text{ berechnen}}}} $$

\paragraph*{Definiere} $\delta_h(x,y)= \begin{cases} 1 & \text{falls } h(x)=h(y) \\ 0 & \text{sonst} \end{cases}$

\paragraph*{Uns interessiert} $\frac{1}{|H|} \cdot \sum\limits_{h\in H} \sum\limits_{y\in S} \delta_h(x,y) = \frac{1}{|H|} \cdot \sum\limits_{y\in S} \underbrace{\sum\limits_{h\in H} \delta_H(x,y)}_{*}$ 
\note{* Anzahl der Hashfunktionen $h \in H$, die $x$ in gleichen Hashtafeleintrag mappen wie $y$; falls $x\not=y$ sonst: |H|}
$$ \leq\footnote{da c-universell} \sum\limits_{y\in S}[\text{falls } x=y:1|\text{ sonst: } \frac{c}{m}] \leq \begin{cases} x\in S & 1+\frac{c(n-1)}{m}\\ x\not\in S & \frac{c}{m} \cdot n \end{cases} \leq 1+c \cdot B $$

\par\medskip
Noch besser wäre nicht nur erwartete Zugriffszeit $O(1)$, sondern Worst-Case Zugriffszeit $O(1)$\par\medskip

$\Rightarrow$ Perfektes Hashing\par\medskip