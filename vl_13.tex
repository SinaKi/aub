\section{Mehrband-Turingmaschine (mit variablen Köpfen)}

\paragraph*{Def. (k-Band TM)} Eine k-Band Turingmaschine ist eine Verallgemeinerung der Turingmaschine, welche über $k$ Speicherbänder mit jeweils unabhängigem Kopf verfügt. Die Zustandsübergangsfunktion hat die Form: $$ \delta : (Q\backslash\{\overline{q}\}) \times \Gamma^k \rightarrow Q \times \Gamma^k \times \{L,RlN\}^k $$ Hierbei ist Band 1 Ein-/Ausgabe, die Bänder $2,\dots,k$ sind initial mit lauter $B$s beschrieben.

\paragraph*{Satz} Eine k-Band Turingmaschine, die mit Rechenzeit $t(n)$ und platz $s(n)$ auskommt, kann von einer 1-Band Turingmaschine $M'$ mit Zeitbedarf $O(t^2(n))$ und Platzbedarf $O(s(n))$ simuliert werden.

\paragraph*{Beweis Idee} k-Band Turingmaschine %TODO bild
Simulation eines Übergangs der k-Band Turingmaschine erfolgt durch 'Sweep' von links nach rechts um alle Zeichen unter den S/L-Köpfen aufzusammeln. In einem Sweep von rechts nach links werden die entsprechenden Zeichen geschrieben / Köpfe bewegt.

\para{Sweeps dauern $O(t(n))$, da nicht mehr als $t(n)$ Positionen der k-Band Turingmaschine beschrieben sein können.}

\section{Die Universelle Turingmaschine}

Die universelle Turingmaschine $U$ simuliert jede andere Turingmaschine $M$ auf beliebiger Eingabe $w$. Als Eingabe erhält$U$ einen String der Form $$<M>w$$ $<M>$ heißt auch Gödelnummer. Als Gödelnummerierung bezeichnet man eine injektive Abbildung von der Menge aller Turingmaschinen in die Menge $\{0,1\}^*$ %TODO notiz

\subsection{'Unsere Gödelnummerierung'}

\paragraph*{Annahme} über unsere Turingmaschine $M$, die zu kodieren ist.
\begin{itemize}
	\item $Q=\{q_1,q_2.\dots,q_t\}$
	\item Anfangszustand ist immer $q_1$, Stoppzustand ist $q_2$
	\item Bandalphabet ist $\Gamma=\{0,1,B\}=\{x_1,x_2,x_3\}$
	\item Die Kopfbewegungen $\{L,R,N\}$ werden nummeriert als $\{D_1,D_2,D_3\}$
\end{itemize}
Die Turingmaschine $M$ wird nun kodiert als Binärkodierung der Übergangsfunktion. Der Übergang $\delta(q_i,X_j)=(q_k,X_l,D_m)$ wird kodiert als $$0^i10^j10^k10^l10^m$$ Wir bezeichnen die Kodierung des t-ten Übergangs als $code(t)$. Die Gödelnummer einer Turingmaschine $M$ mit $s$ übergängen ist somit $$code(1)11code(2)11code(3)11 \dots 11code(s)111$$ %TODO notizen

\para{Die universelle Turingmaschine $U$ erhält als Eingabe $<M>w$. Wir implementieren $U$ als 3-Band Turingmaschine:
\begin{itemize}
	\item Check ob $<M>$ korrekte Kodierung einer Turingmaschine hat (falls nein Fehlerausgabe)
	\item kopiere $<M>$ auf Band 2
	\item lösche $<M>$ von Band 1, sodass nur noch $w$ dort steht
\end{itemize}}

\paragraph*{In Simulation} enthält Band \dots
\begin{itemize}
	\item 1 den Bandinhalt von $M$
	\item 2 die Beschreibung von $M$
	\item 3 den Zustand von $M$
\end{itemize}

\paragraph*{Ein Schritt von $M$}
\begin{itemize}
	\item Suche zu Zeichen, das auf Band 1 gerade gelesen wird und Zustand auf Band 3, den Übergang auf Band 2, der 'passt'
	\item dann: aktualisiere entsprechend Band 1
	\item \dots bewege Kopf auf Band 1
	\item \dots Schreibe neuen Zustand auf Band 3
\end{itemize}

Wir gehen davon aus, dass $<M>$ konstant groß ist, d.h. wir können einen Schritt von $M$ in $O(|<M>|)=O(1)$ simulieren. Wenn wir die als 3-Band Turingmaschine implementierte universelle Turingmaschine $U$ als 1-Band Turingmaschine implementieren wollen, geht das mit quadratischem Zeitoverhead, oder besser.

\section{Die Registermaschine (RAM)}
Die RAM ist ein Maschinenmodell, welches assemblerähnliche Programmierung erlaubt. Der Speicher der RAM ist unbeschränkt und besteht aus Registern $c(0),c(1),c(2),\dots$ Der $c(0)$ ist besonders und heißt Akkumulator. \textbf{Inhalte der Register sind beliebig große ganze Zahlen.}

\para{Des Weiteren enthält die RAM einen Befehlszähler $b$ (initial auf 1) %TODO bild
Ein Programm für die RAM ist endlich, in jedem Schritt wird die Zeile auf die $b$ zeigt abgearbeitet. Die Eingabe der RAM steht zu Beginn in $c(1),c(2),\dots,c(k)$ für $k \in \mathbb{N}$}

\para{Die Ausgabe steht in $c(1)\dots c(l), l \in \mathbb{N}$ unter der Annahme, dass $l,k$ fix, berechnet RAM eine Funktion: $$\delta:\mathbb{Z}^k \rightarrow \mathbb{Z}^l \cup \{ \perp \}$$} %TODO notiz

\para{Syntax und Bedeutung einiger RAM-Befehle:}

\begin{table}[htb!]
\centering
\begin{tabular}{l|l|l}
GOTO j & & $b=j$ \\
\hline
IF(c(0)=x) GOTO j & & $b := \begin{cases}j & \text{ falls } c(0)=x \\ b+1 & \text{sonst}\end{cases}$ \\
\hline
LOAD i & $c(0)=c(i)$ & $b:=b+1$ \\
CLOAD i & $c(0)=i$ & $b:=b+1$ \\
INLOAD i & $c(0)=c(c(i))$ & $b:=b+1$ \\
\hline
STORE i & $c(i)=c(0)$ & $b:=b+1$ \\
\hline
ADD i & $c(0)=c(0)+c(i)$ & $b:=b+1$ \\
CADD i & $c(0)=c(0)+i$ & $b:=b+1$ \\
\end{tabular}
\end{table}

Offensichtlich ist die RAM mindestens so mächtig wie Rechner, die wir kennen. Zeitverbrauch einer Rechnung auf der RAM könnte sein
\begin{itemize}
	\item jeder Schritt zählt 1 (uniformes Kostenmaß)
	\item ein Schritt kostet proportional zur binären Kodierungslänge der auftretenden Zahl (logarithmisches Kostenmaß)
\end{itemize}