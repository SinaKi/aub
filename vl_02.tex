\section{Clostest Pair Problem ($n$ Punkte im $\mathbb{R}^2$)}
\begin{itemize}
	\item Deterministisch: $O(n \log n)$ z.B. mittels D\&C
	\item Randomisiert: erwartete Laufzeit: $O(n)$
\end{itemize}
\subsection{Randomisiert incrementeller Algoithmus}
\begin{itemize}
	\item bringe die Punktmenge in eine zufällige Reihenfolge $p_1,\dots,p_n$
	\item invariante
	\begin{itemize}
		\item $\delta_i :=$ Minimalabtand zweier Punkte in $\{ p_1,\dots,p_i \}$
		\item haben immer Gitterstruktur mit Maschenweite $\delta_i$, in welcher $p_1,\dots,p_i$ eingefügt sind
	\end{itemize}
	\item wichtigste Operation: Hinzunahme von $p_{i+1}$ unter Aufrechterhaltung der Instanzen:
	\begin{itemize}
		\item lokalisieren von $p_{i+1}$ ?? akt. Gitterstruktur | $O(1)$
		\item Untersuchung der $O(1)$ Punkte in den benachbarten Gitterzellen der Zelle von $p_{i+1}$ auf ?? für neues closest pair | $O(1)$
	\end{itemize}
	\item[] Fall A: $\delta_0 = \delta_{i+1} \Rightarrow$ füge $p_{i+1}$ in Gitterstruktur ein | $O(1)$
	\item[] Fall B: $\delta_i > \delta_{i+1} \Rightarrow$ baue neues Gitter mit Maschenweite $\delta_{i+1}$, füge $p_1,\dots,p_{i+1}$ ein | $O(i)$
\end{itemize}

Schlechte Punktkonfiguration und Reihenfolge führt zu $\Theta(a^2)$ Laufzeit.

\paragraph*{Idee:} Betrachte Punkte in zufälliger Reihenfolge $\Rightarrow$ Hoffnung, Fall B tritt seltener auf.

Die erwartete Laufzeit des Algorithmus ist
\[ \text{Laufzeit} = \sum_{i=3}^n \text{Kosten für Einfügen von Punkt } p_i \]

\[\text{Erwartete Laufzeit }= E(\sum_{i=3}^n Kosten für Einfügen von Punkt p_i)\]
\[ = \sum_{i=3}^n \underbrace{E(\text{Kosten für Einfügen von Punkt} p_i)}_{\text{Fall A: O(1); Fall B: O(i)}} \]

Fall B tritt ein, wenn $\delta_i < \delta_{i-1}$ d.h. wenn durch Hinzufügen von $p_i$ sich die CP-Distanz ändert. $\delta_i < \delta_{i-1}$ kann nur passen, wenn $p_i$ einer der beiden Punkte aus $\{ p_1,p_2,\dots,p_i \}$ ist, welcher das CP definiert. Seien $p_a,p_b$ zwei Punkte aus $\{ p_1,\dots,p_i \}$. Fall B kann nur eintreten, falls $p_i = p_a$ oder $p_i = p_b$.
Da jeder Punkt aus $\{ p_1,\dots,p_i \}$ mit gleicher Wahrscheinlichkeit der zuletzt hinzugefügt ist, ist diese Wahrscheinlichkeit $\leq \frac{2}{i}$.
\begin{itemize}
	\item[$\Rightarrow$] Erwartete Laufzeit für Einfügen von $p_i$ ist $\leq O(1) + \frac{2}{i} \cdot O(i) = O(1)$
	\item[$\Rightarrow$] erwartete Gesamtlaufzeit $\sum\limits_{i=3}^n O(1) = O(n)$
\end{itemize}

\paragraph*{\underline{Wichtig:}} Die Laufzeit dieses Algorithmus hängt \underline{nicht} von der Eingabe ab, sie ist immer erwartet $O(n)$, egal wie die Eingabe aussieht.

\paragraph*{Anmerkung zur Gitterstruktur:} Angenommen wir haben Gitterstruktur mit Maschenwert $\delta p(p_x,p_y)$ fällt in Gitterzelle $(\frac{p_x}{\lfloor \delta \rfloor},\frac{p_y}{\lfloor \delta \rfloor})$.
\paragraph*{Annahme:} Wir haben eine Datenstruktur, welche in $O(1)$ ein Element in Zelle $(i,j)$ hinzufügen und in $O(1)$ den Inhalt einer Zelle zurückgeben kann. Größe der Datenstruktur soll $O(\text{\# der eingefügten Element})$ sein.

Im obigen Algorithmus hängt nur die Laufzeit vom Zufall ab, das Ergebnis ist immer korrekt. So einen Algorithmus nennt man LAS VEGAS ALGORITHMUS.

\subsection{MONTE CARLO ALGORITHMEN}
\begin{itemize}
	\item[\dots] Laufzeit abhängig vom Zufall
	\item[\dots] Korrektheit des Ergebnisses hängt vom Zufall ab
\end{itemize}

\subsection{Das MinCut-Problem}
Geg.: ungerichteter, ungewichteter Multigraph $G(V,E)$ mit $|V|=n,|E|=m$. Eine Teilmenge $A \subseteq V$ der Knoten induziert einen sogenannten \underline{Cut}.
$$ cut(A,G) = \{ e = \{ v,w \} \in E\ \big|\ |\{ v,w \} \cap A | = 1 \} $$
Der Wert eines cuts ist seine Kardinalität $|cut(A,G)|$
\paragraph*{Das Mincut-Problem für einen Graph $G(V,E)$:} Finde $A \varsubsetneq V$ mit $|cut(A,G)|$ ist minimal.


\begin{figure}[!h]
\centering
\begin{tikzpicture}[->,>=stealth',shorten >=1pt,auto,node distance=2.5cm,
        thick,main node/.style={circle,draw,minimum size=1cm,inner sep=0pt]}]
        
    \draw[blue] (4.7,-0.5) circle (1.8);

    \node[main node] (1) {b};
    \node[main node] (2) [below left = 0.7cm and 1.5cm of 1]  {a};
    \node[main node] (3) [below right = 0.7cm and 1.5cm of 1] {e};
    \node[main node] (4) [below left = 2.4cm and 0.5cm of 1] {c};
    \node[main node] (5) [below right = 2.4cm and 0.5cm of 1] {d};
    \node[main node] (6) [right = 2.5cm of 1] {f};
    \node[main node] (7) [right = 1.4cm of 6] {h};
    \node[main node] (8) [below right = 1cm and 0.5cm of 6] {g};

    \path[-]
    (1) edge node {} (2)
        edge node {} (3)
        edge node {} (4)
        edge node {} (5)
    (2) edge node {} (4)
    (3) edge node {} (5)
    (4) edge node {} (5)
    (6) edge node {} (1)
    	edge node {} (7)
    	edge node {} (8)
    (7) edge node {} (8);
\end{tikzpicture}
MinCut mit Wert 1.
\end{figure}

\begin{figure}[!h]
\centering
\begin{tikzpicture}[->,>=stealth',shorten >=1pt,auto,node distance=2.5cm,
        thick,main node/.style={circle,draw,minimum size=1cm,inner sep=0pt]}]
        
    \draw[blue] (1.7,-1.4) circle (1);

    \node[main node] (1) {b};
    \node[main node] (2) [below left = 0.7cm and 1cm of 1]  {a};
    \node[main node] (3) [below right = 0.7cm and 1cm of 1] {c};
    
    \path[-]
    (1) edge node {} (2);
\end{tikzpicture}
MinCut mit Wert 0.
\end{figure}

\paragraph*{Anmerkung:} Minimalgrad eines Knotens in $G$ ist immer obere Schranke für den Wert des MinCuts.
\paragraph*{Anwendungen:}
\begin{itemize}
	\item Netzwerke und Fehlertoleranz
	\item Internet, Kommunikationsnetz, \dots
	\item Strom
	\item Abwasser
\end{itemize}
Wie können wir den MinCut naiv berechnen?
\paragraph*{Idee:} Teste alle möglichen Partitionen von $V$ $$ \sum_{i=1}^{n-1} {n \choose i} = 2^n - 2 \hspace{1cm}\text{für } n = |V|$$
Nicht praktikabel, da schon für einen Graphen mit 128 Knoten mehr Partitionen möglich sind als es Atome im Universum gibt.

\subsubsection{Kargers MinCut Algorithmus}
Randomisierter Algorithmus, der mit Wahrscheinlichkeit $~\frac{1}{n^2}$ das richtige Ergebnis berechnet. Die zentrale Operation des Algorithmus ist die \textbf{Kantentraktion}.
\begin{enumerate}
	\item wähle eine Kante $\{ v,w \}$ aus
	\item ersetzte Knoten $v$ und Knoten $w$ durch Knoten $vw$, der alle Kanten von $v$ und $w$ erbt (ohne Schlingen).
\end{enumerate}

\begin{figure}[!h]
\centering
\begin{tikzpicture}[->,>=stealth',shorten >=1pt,auto,node distance=2.5cm,
        thick,main node/.style={circle,draw,minimum size=0.7cm,inner sep=0pt]}]

	\node at (5,0) {$\Rightarrow$};

    \node[main node] (1) {v};
    \node[main node] (2) [right = 2cm of 1] {w};
    \node[main node] (3) [below = 1cm of 1,minimum size=0.3cm] {};
    \node[main node] (4) [below left = 0.7cm and 1cm of 1,minimum size=0.3cm] {};
    \node[main node] (5) [above left = 0.7cm and 1cm of 1,minimum size=0.3cm] {};
    \node[main node] (6) [above right = 0.7cm and 1cm of 1,minimum size=0.3cm] {};
    \node[main node] (7) [below = 1cm of 2,minimum size=0.3cm] {};
    \node[main node] (8) [below right = 0.7cm and 1cm of 2,minimum size=0.3cm] {};
    \node[main node] (9) [above right = 0.7cm and 1cm of 2,minimum size=0.3cm] {};

    \node[main node] (10) [right = 7cm of 1] {vw};
    \node[main node] (11) [below left = 1cm and 0.5cm of 10,minimum size=0.3cm] {};
    \node[main node] (12) [above left = 1cm and 0.5cm of 10,minimum size=0.3cm] {};
    \node[main node] (13) [below right = 1cm and 0.5cm of 10,minimum size=0.3cm] {};
    \node[main node] (14) [above right = 1cm and 0.5cm of 10,minimum size=0.3cm] {};
    \node[main node] (15) [left = 1cm of 10,minimum size=0.3cm] {};
    \node[main node] (16) [below right = 0.4cm and 1cm of 10,minimum size=0.3cm] {};
    \node[main node] (17) [above right = 0.4cm and 1cm of 10,minimum size=0.3cm] {};
    
    \path[-]
    (1) edge[bend left] (2)
    	edge[bend right] (2)
    	edge[bend left] (3)
    	edge[bend right] (3)
    	edge (4)
    	edge (5)
    	edge (6)
    (2) edge (6)
    	edge (3)
    	edge (7)
    	edge (8)
    	edge (9)
    (8) edge (9)
    (10)edge (11)
    	edge[bend left] (11)
    	edge[bend right] (11)
    	edge (12)
    	edge (13)
    	edge[bend left] (14)
    	edge[bend right] (14)
    	edge (15)
    	edge (16)
    	edge (17)
    (17)edge (16);
\end{tikzpicture}
\end{figure}

\begin{lstlisting}
for 1 to n-2
	Kontrahiere zufaellige Kante
Gib Knotenmenge entsprechend einer der beiden verbleibenden Knoten aus
\end{lstlisting}

\paragraph*{Annahme:} Es gibt genau einen MinCut und dessen Wert ist $K$.

\paragraph*{Beobachtung:} Algorithmus berechnet den MinCut $\Leftrightarrow$ in keiner der $n-2$ Kontraktionen wurde eine der $K$ MinCut-Kanten Kontrahiert.

\newpage
\begin{figure}[!h]
\centering
\begin{tikzpicture}[->,>=stealth',shorten >=1pt,auto,node distance=2.5cm,
        thick,main node/.style={circle,draw,minimum size=0.7cm,inner sep=0pt]}]

	\node at (6,-0.7) {ac};
	\node at (6,-1) {$\Rightarrow$};

    \node[main node] (1) {a};
    \node[main node] (2) [below = 1cm of 1] {b};
    \node[main node] (3) [below right = 0.4cm and 1cm of 1] {c};
    \node[main node] (4) [right = 1cm of 3] {d};
    \node[main node] (5) [above right = 0.4cm and 1cm of 4] {e};
    \node[main node] (6) [below right = 0.4cm and 1cm of 4] {f};

    \node[main node] (7) [right = 1.8cm of 6] {b};
    \node[main node] (8) [above right = 0.7cm and 0.7cm of 7] {ac};
    \node[main node] (9) [right = 1cm of 8] {d};
    \node[main node] (10) [above right = 0.4cm and 1cm of 9] {e};
    \node[main node] (11) [below right = 0.4cm and 1cm of 9] {f};
    
    \path[-]
    (1) edge (2)
    	edge (3)
    (2) edge (3)
    (3) edge (4)
    (4) edge (5)
    	edge (6)
    (5) edge (6)
    
    (7) edge [bend left] (8)
    	edge [bend right] (8)
    (8) edge (9)
    (9) edge (10)
    	edge (11)
    (10) edge (11);
\end{tikzpicture}
\end{figure}

\begin{figure}[!h]
\centering
\begin{tikzpicture}[->,>=stealth',shorten >=1pt,auto,node distance=2.5cm,
        thick,main node/.style={circle,draw,minimum size=0.7cm,inner sep=0pt]}]

	\node at (-1,1.3) {ac-d};
	\node at (-1,1) {$\Rightarrow$};
	\node at (4.5,1.3) {e-f};
	\node at (4.5,1) {$\Rightarrow$};

    \node[main node] (1) {b};
    \node[main node] (2) [above right = 0.7cm and 0.7cm of 1] {acd};
    \node[main node] (3) [above right = 0.4cm and 1cm of 2] {e};
    \node[main node] (4) [below right = 0.4cm and 1cm of 2] {f};

    \node[main node] (5) [right = 5cm of 1] {b};
    \node[main node] (6) [above right = 0.7cm and 0.7cm of 5] {acd};
    \node[main node] (7) [right = 1cm of 6] {ef};
    
    \path[-]
    (2) edge [bend left] (1)
    	edge [bend right] (1)
    	edge (3)
    	edge (4)
    (3) edge (4)
    
    (6) edge [bend left] (5)
    	edge [bend right] (5)
    	edge [bend left](7)
    	edge [bend right](7);
\end{tikzpicture}
\end{figure}

\begin{figure}[!h]
\centering
\begin{tikzpicture}[->,>=stealth',shorten >=1pt,auto,node distance=2.5cm,
        thick,main node/.style={circle,draw,minimum size=0.7cm,inner sep=0pt]}]

	\node at (-1,1.3) {b-acd};
	\node at (-1,1) {$\Rightarrow$};

    \node[main node] (1) {abcd};
    \node[main node] (2) [right = 1cm of 1] {ef};
    
    \path[-]
    (1) edge [bend left] (2)
    	edge [bend right] (2);
\end{tikzpicture}
\end{figure}
Qutput $= \{ e,f \}$ induziert Cut mit Wert 2.

%TODO graphen, notiz