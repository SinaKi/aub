\paragraph*{Beweisidee} Angenommen wir hätten eine Turingmaschine $M_H$, die das Halteproblem entscheidet, könnte $\overline{D}$ entscheiden (und das wissen wir, geht nicht). $M_H$ wird im Beweis als 'Unterprogramm' einer Turingmaschine aufgerufen welche $\overline{D}$ angeblich entscheidet.

\para{} Jetzt betrachten wir das spezielle Halteproblem, d.h. gegeben eine Kodierung einer Turingmaschine $M$
hält $M$ auf dem leeren Wort? Die zu entscheidende Sprache ist somit $$ H_E = \{ <M> | M \text{ hält auf Ergebnis }  E \} $$

\paragraph*{Satz} Das spezielle Halteproblem $H_E$ ist nicht Turing-entscheidbar.

\paragraph*{Beweis} Angenommen wir hätten eine Turingmaschine $M_E$ welche $H_E$ entscheidet. Wir bauen eine Turingmaschine $M_H$, welche $H$ entscheidet und dabei $M_E$ als Unterprogramm benutzt. Da es $M_H$ nicht geben kann, kann es $M_S$ auch nicht geben. $M_H$ arbeitet wie folgt auf Eingabe $<M>w$:
\begin{enumerate}
	\item Falls $<M>$ keine gültige TM-Kodierung, verwerfe
	\item Andernfalls konstruiere eine Kodierung einer Turingmaschine $M_w^*$ mit folgenden Eigenschaften:
	\begin{itemize}
		\item Falls $M_w^*$ mit der leeren Eingabe startet schreibt sie $w$ aufs Band und simuliert die Turingmaschine $M$ auf $w$
		\item ansonsten macht sie, was sie will
	\end{itemize}
	\item Frage die Turingmaschine $M_\epsilon$ ob die Turingmaschine $M_w^*$ auf leerer Eingabe terminiert; übernehme Antwort von $M_\epsilon$
\end{enumerate}

\para{} Diese Konstruktion ist korrekt, da \note{$<?>$ ist die Kodierung von ?}
\begin{itemize}
	\item[$<M>$] $w \in H$
	\item[$\Rightarrow$] $M$ hält auf $w$
	\item[$\Rightarrow$] $M_\epsilon^*$ hält auf $\epsilon$
	\item[$\Rightarrow$] $M_\epsilon$ akzeptiert $<M_\epsilon^*>$
	\item[$\Rightarrow$] $M_H$ akzeptiert $<M>w$
	
	\item[$<M>$] $w \not\in H$
	\item[$\Rightarrow$] $<M>$ hält nicht auf $w$
	\item[$\Rightarrow$] $M_\epsilon^*$ hält nicht auf $\epsilon$
	\item[$\Rightarrow$] $M_\epsilon$ verwirft $M_\epsilon^*$
	\item[$\Rightarrow$] $M_H$ akzeptiert $<M>w$ \hspace{1cm} $\Box$
\end{itemize}


\section{Der Satz von RICE}

Turingmaschinen berechnen partielle Funktionen, da sie nicht auf jeder Eingabe halten, d.h. man könnte schreiben $$ f_M\{0,1\}^* \rightarrow \{0,1\}^*\cup\{\perp\} $$ Hierbei steht $\{\perp\}$ für 'nicht definiert' und bedeutet, dass die Maschine nicht terminiert. Für Entscheidungsprobleme haben wir $$ f_M=\{0,1\}^* \rightarrow \{0,1\}\cup\{\perp\}$$ Hierbei steht $'0'$ für akzeptieren und $'1'$ für verwerfen. $$ R=\{  f_M:\{0,\}^* \rightarrow \{0,1\}^*\cup\{\perp\}|M \text{ ist eine Turingmaschine}\} $$ \dots also die Menge der von Turingmaschinen berechenbaren Funktionen.

\paragraph*{Satz} Sei $S$ eine Teilmenge von $R$ mit $\emptyset\not=S\not=R$. Dann ist die Sprache $L(S)=\{<M>|M$ berechnet eine Funktion aus $S\}$ nicht Turing-entscheidbar.

\paragraph*{Beweis} Wieder mittels Unterprogrammtechnik. Aus einer angeblichen Turingmaschine $M_{L(S)}$, die $L(S)$ entscheidet konstruieren wir eine Turingmaschine $M_\epsilon$, welche das spezielle Halteproblem $H_\epsilon$ entscheidet, das stünde im Widerspruch zur Unentscheidbarkeit von $H_\epsilon$. Sei $u$ die überall undefinierte Funktion. Für gegebenes $S$ gibt es zwei Fälle.
\begin{itemize}
	\item[1)] $u \not\in S$. Sei $f \not= u$ eine Funktion aus $S$ (existiert, da sonst $S$ leer). Die Turingmaschine $M_E$ funktioniert wie folgt:
	\begin{enumerate}
		\item Falls Eingabe keine Turingmaschine kodiert, verwerfe
		\item Konstruiere eine Turingmaschine $M^*$ mit folgenden Eigenschaften:
		\begin{enumerate}
			\item ignoriere Eingabe $x$. Simuliere das Verhalten von $M$ bei Eingabe $\epsilon$ auf einer separaten Spur
			\item Berechne $f(x)$, d.h. Simuliere $M_f$ auf Eingabe $x$.
		\end{enumerate}
		\item Benutze $M_{L(S)}$ als Unterprogramm um zu entscheiden, ob $M^*$ eine Funktion aus $S$ berechnet.
	\end{enumerate}
	\item[] Korrektheit folgt, da für Eingaben der Form $w=<M>$ gilt:
	\begin{itemize}
		\item[$w\in  H_\epsilon$] $\Rightarrow M$ hält auf $\epsilon$
		\item[] $\Rightarrow M^*$ berechnet $f$
		\item[] $\Rightarrow <M^*> \in L(S)$
		\item[] $\Rightarrow M_{L(S)}$ akzeptiert $<M^*>$
		\item[] $\Rightarrow M_\epsilon$ akzeptiert $w$
		\item[$w\not\in  H_\epsilon$] $\Rightarrow M$ hält nicht auf $\epsilon$
		\item[] $\Rightarrow M^*$ berechnet $u$
		\item[] $\Rightarrow <M^*> \in L(S)$
		\item[] $\Rightarrow M_{L(S)}$ verwirft $<M^*>$
		\item[] $\Rightarrow M_\epsilon$ verwirft $w$
	\end{itemize}
	\item[2)] $u \in S$. Dann wähle $f$ aus $R \backslash S \not= \emptyset$ und invertiere Akzeptanzverhalten. $\Box$
\end{itemize}

\para{} Der Satz von RICE schließt bspw. aus, dass es Compiler gibt, die apriori Terminierung und Korrektheit von Programmen entscheiden können.


\section{Semi-Entscheidbarkeit und Rekursive Aufzählbarkeit}

\paragraph*{Def.} Eine Sprache $L$ wird von einer Turingmaschine $M$ entschieden, wenn $M$ auf jeder Eingabe hält und genau die Wörter aus $L$ akzeptiert.

\note{Eine Sprache ist entscheidbar, wenn es eine Turingmaschine $M$ gibt, welche $L$ entscheidet.}

\paragraph*{Def.} Eine Sprache $L$ wird von einer Turingmaschine $M$ erkannt, wenn $M$ jedes Wort aus $L$ akzeptiert und kein Wort aus $\mathcal{E}^*\backslash L$ akzeptiert. Auf Eingabe nicht aus $L$ muss $M$ nicht halten.

\paragraph*{Def.} Eine Sprache heißt semi-entscheidbar, falls es eine Turingmaschine $M$ gibt, welche $L$ erkennt.

\paragraph*{Bsp.} $H_\epsilon$ ist nicht entscheidbar, aber semi-entscheidbar.

\subsection{Alternative Charakterisierung der Semi-entscheidbaren Sprachen.}

Ein Aufzähler einer Sprache $L$ ist eine spezielle Turingmaschine $M$, welche einen 'Drucker' hat, d.h. ein Ausgabeband auf dem der Kopf nur nach rechts bewegt werden kann. Ein Aufzähler gibt (gestartet auf leerem Band) alle Worte aus $L$ auf dem Drucker aus, also
\begin{itemize}
	\item der Aufzähler gibt nur Wörter aus $L$ aus
	\item jedes Wort aus $L$ wird irgendwann gedruckt
\end{itemize}

\paragraph*{Def.} Ein Sprache $L$ heißt rekursiv aufzählbar, wenn es einen Aufzähler für $L$ gibt.

\paragraph*{Satz} Eine Sprache $L$ ist genau dann Semi-entscheidbar, wenn $L$ rekursiv aufzählbar ist.

\paragraph*{Beweis} 
\begin{itemize}
	\item[$\Leftarrow'$] Annahme, wir haben einen Aufzähler für $L$ und konstruieren eine Turingmaschine, welche $L$ erkennt. $M$ lässt den Drucker immer ein Wort ausdrucken und akzeptiert das erkennende Wort das dem ausgedruckten entspricht. Andernfalls lasse nächstes Wort ausdrucken.
	\item[$'\Rightarrow'$] Annahme, wir haben eine Turingmaschine $M$, welche $L$ erkennt, wir bauen daraus eine Turingmaschine $M'$, welche $L$ aufzählt. Seien $w_1,w_2,w_3,\dots$ die Wörter aus $\sum^*$ in kanonischer Reihenfolge. Der Aufzähler arbeitet wie folgt:
	\begin{itemize}
		\item[] Für $I=1,2,3,\dots$
		\item simuliere $i$ Schritte des Erkenners $M$ auf allen Wörtern $w_1,w_2,\dots,w_i$
		\item falls Wort dabei akzeptiert wird, drucke es aus
	\end{itemize}
	\item[] Falls das Wort $w_K \subset L$ und der Erkenner $t_K$ Schritte zum akzeptieren bracht, wird das Wort in der $max(K,t_K)$-ten Runde gedruckt (und danach auch immer). $\Box$
\end{itemize}