\section{Berechenbarkeit}

\para{Uns interessieren Fragestellungen der Art: Was kann man überhaupt berechnen? Was sind geeignete Rechenmodelle?}

\paragraph*{Etwas Notation} endliches Alphabet $\sum i$ oft: $\sum = \{ 0,1 \}$ Die Menge aller Würter der Länge $k$ über $\sum$ bezeichnen wir als $\sum\limits^k$, z.B. \note{kanonische oder lexikographisch Reihenfolge} $$ \{ 0,1 \}^3 = \{ 000,001,010,011,100,101,110,111 \} $$

\paragraph*{Probleme} $\overset{\wedge}{=}$ 'Aufgaben' oder 'Fragestellungen', die wir mit dem Rechner lösen/beantworten wollen.


\subsection*{Problem 1.1}
\paragraph*{Eingabe} eine binär kodierte Zahl $q \in \mathbb{N}$; $q \geq 2$

\paragraph*{Ausgabe} ein binär kodierter Primfaktor von $q$

\paragraph*{Bsp.} Eingabe: 110 (die Zahl 6); Ausgabe: 010 (die Zahl 2) oder 011 (die Zahl 3)

%TODO